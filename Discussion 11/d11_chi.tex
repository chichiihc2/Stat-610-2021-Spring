\documentclass[12pt]{extarticle}
\usepackage{amsmath}
\usepackage{amssymb}
\usepackage{amsfonts, bm}
\usepackage{graphicx}
\usepackage{fancyhdr}
\usepackage{multirow}
\usepackage[hmargin={0.8in, 0.8in}, vmargin={1.0in, 1.0in}]{geometry}
\thispagestyle{fancy}
\pagestyle{fancy}
\fancyhead{}
\fancyfoot{}
\newcommand{\p}{\mathbb P}
\newcommand{\x}{\mathbf x}
\newcommand{\X}{\mathbf X}
\newcommand{\E}{\mathbb E}
\newcommand{\N}{\mathcal N}
\lhead{\small cdai39@wisc.edu} \chead{TA: Chi-Shian Dai} 

%-------------------------------------%
\begin{document}
	
	\begin{center}
		{\large \bf STAT 610: Discussion 11}
	\end{center}
	\vspace{0.22cm}
%-------------------------------------%
----------------------------------%


  

\section{Summary}
\begin{itemize}
	\item If we have a pivot $Q(X,\theta)$, a $1-\alpha$ confidence interval involves finding $a$ and $b$ so that $\p(a<Q<b) = 1-\alpha$. Typically, the length of the interval on $\theta$ will be some function of $a$ and $b$ like $b-a$ or $1/a^2 - 1/b^2$. If $Q$ has density $f$ and the length can be expressed as $\int_a^bg(t)dt$, the shortest pivotal interval is the solution of 
	$$\min_{C}\int_C g(t)dt \hspace{3mm} \text{subject to}  \hspace{3mm} \int_C f(t)dt = 1-\alpha.$$
	Then, the solution of the above constraint optimization problem is $C = \{t: g(t)<\lambda f(t)\}$, where $\lambda$ is chosen so that $\int_C f(t)dt = 1-\alpha$.
	\item UMA confidence set: A $1 - \alpha$ set is \textit{uniformly most accurate} (UMA) if it minimizes the probability of false coverage over a class of $1-\alpha$ confidence sets.
	\begin{itemize}
		\item UMA confidence sets are constructed by inverting the acceptance region of UMP tests.
	\end{itemize}
\end{itemize}

\section{Questions}
 
\begin{enumerate}

	
	\item Let $X_1,\ldots, X_n$ be i.i.d Exp($\theta$). 
	\begin{enumerate}
		\item Find a UMP size $\alpha$ hypothesis test of $H_0: \theta = \theta_0$ v.s. $H_1: \theta < \theta_0$.
		\item Find a UMA $1 - \alpha$ confidence interval based on inverting the test in part (a). Show that the interval can be expressed as $$C^*(x_1,\ldots, x_n) = \Bigg\{\theta: 0\leq \theta\leq \dfrac{\sum x_i}{G_{n,\alpha}}\Bigg\},$$
		where $G_{n,\alpha}$ is $\alpha$ quantile of $Gamma(n,1).$
		\item Find the expected length of $C^*(x_1,\ldots, x_n)$.
	\end{enumerate}

	\vspace{6cm}
	
	\item Let $X_1,\dots,X_n$ be i.i.d. from the distribution $E(\theta,\theta)$, where $\theta>0$ is unknown.
	\begin{enumerate}
	\item Show that both $\bar{X}/\theta$, and $X_{(1)}/\theta$ are pivotal quantities.
	\item  Find the 1-$\alpha$ confidence intervals based on these two pivotal quantities.
	\item Which one is better?
	\end{enumerate}
\vspace{8cm}

\item \textit{(Cox's Paradox)} We are to test $$H_0:\theta = \theta_0\hspace{3mm}\text{versus}\hspace{3mm} H_1:\theta>\theta_0,$$
where we observe $X$ with distribution 
$$X\sim \begin{cases}
\mathcal N(\theta,100),& \text{with probability } p\\
\mathcal N(\theta, 1),              & \text{with probability } 1-p.
\end{cases}$$
\begin{enumerate}
	\item Show that the test given by \[\text{reject }H_0 \text{ if }X>\theta_0+z_\alpha\sigma, \]
	where $\sigma = 1$ or 10 depending on which population is sampled, is a level $\alpha$ test.
	\item Show that the following test of size $\alpha$ is given by 
	\[\text{reject }H_0 \text{ if }X>\theta_0+z_{(\alpha-p)/(1-p)} \text{ and }\sigma=1;\text{ otherwise always reject} H_0.\]
	Derive a $1-\alpha$ confidence set by inverting the acceptance region of this test, and show that it is the empty set with positive probability.
\end{enumerate}
\end{enumerate}
%-------------------------------------%
\end{document}
